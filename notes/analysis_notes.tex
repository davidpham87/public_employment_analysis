% Created 2016-01-14 Don 00:23
\documentclass[a4paper]{article}
\usepackage[utf8]{inputenc}
\usepackage[T1]{fontenc}
\usepackage{fixltx2e}
\usepackage{graphicx}
\usepackage{longtable}
\usepackage{float}
\usepackage{wrapfig}
\usepackage{rotating}
\usepackage[normalem]{ulem}
\usepackage{amsmath}
\usepackage{textcomp}
\usepackage{marvosym}
\usepackage{wasysym}
\usepackage{amssymb}
\usepackage{hyperref}
\tolerance=1000
\usepackage[margin=3.5cm]{geometry}
\bibliographystyle{chicago}% ---> Hampel,F., E.Ronchetti,... W.Stahel(1986) ...
\bibliography{biblio.bib}
\author{David Pham}
\date{\textit{<2016-01-13 Mit>}}
\title{Analysis of Public Employment Rate}
\hypersetup{
  pdfkeywords={},
  pdfsubject={},
  pdfcreator={Emacs 24.5.1 (Org mode 8.2.10)}}
\begin{document}

\maketitle
\tableofcontents


\section{Summary of the analysis}
\label{sec-1}

\section{Model and Data}
\label{sec-2}

Following \cite{aaskoven2015fiscal}, the following variable were selected from
the Economic Outlook database to explain the public employment rate:

\begin{center}
\begin{tabular}{ll}
Variable & Oecd code\\
\hline
country & country\\
government employment & eg\\
total employment & et\\
GDP growth & gdpv$_{\text{annpct}}$\\
unemployment rate & unr\\
government spending & ypgtq\\
adult population & pop1574\\
Net household dispossable income & ydrh\\
\hline
\end{tabular}
\end{center}

Additionally, the following time series were retrieved to compare result of the
regression with \cite{aaskoven2015fiscal} and \cite{alesina2000redistributive}:

\begin{center}
\begin{tabular}{ll}
Data & Source\\
\hline
GPD per capita & OECD\\
gini index & OECD\\
gini index & \cite{toth2014chapter}\\
Fiscal Transparency & IMF \cite{wang2015trends}\\
Laassen Fiscal Transparency Score & \cite{aaskoven2015fiscal}\\
\hline
\end{tabular}
\end{center}

Unfortunately, these data had an annual frequency only.

\subsection{Data}
\label{sec-2-1}

\subsubsection{OECD economic outlook}
\label{sec-2-1-1}

Economic outlook data set with annual and quarterly frequency.

\begin{itemize}
\item Governemnt (public) employment: "eg"
\item Total Employement : "et"
\item GDP growth: gdpv$\backslash$$_{\text{annpct}}$
\item unemployment rate: unr
\item ecsa, ee?
\item Time effect: "year" or "date"
\item Government spending "ypgtq", \emph{Total disbursements, general
government, as a percentage of GDP}

\item "POP1574"
\item Electoral incentives (number of electoral years)
\item Economics conditions: ? See if growth or recession
\item country: "country"
\item money income per capita (Net/Gross household disposable income, real:
YDRH($\backslash$$_{\text{G}}$))

\item Ethnicity?
\item Left or right party
\item Portion of population having a university degree
\end{itemize}

\subsection{OECD Population}
\label{sec-2-2}

Population are taken from
\href{http://stats.oecd.org/Index.aspx?DatasetCode=POP_FIVE_HIST}{OECD}.

\begin{itemize}
\item log of city population ?
\end{itemize}

\subsection{National Accounts}
\label{sec-2-3}

\begin{itemize}
\item GDP per capita in constant price. [@oecd2015gdp]
\end{itemize}

\subsection{Inequality}
\label{sec-2-4}

\begin{itemize}
\item Measure of inequality: gini [@oecd2015inequality]
\end{itemize}

\subsection{Unfound data}
\label{sec-2-5}

\begin{itemize}
\item fiscal rules, fiscal transparency
\item data from Lassen (2010 19-20)
\item Fiscal transparency is interacted with gdp growth as it should
contribute as a good factor.
\end{itemize}

\begin{center}
\begin{tabular}{lr}
Country & Fiscal Score\\
\hline
Greece & 1\\
Italy & 2\\
Norway & 2\\
Belgium & 3\\
Denmark & 3\\
Germany & 3\\
Ireland & 3\\
Spain & 3\\
Switzerland & 3\\
Austria & 4\\
France & 4\\
Portugal & 4\\
Canada & 5\\
Finland & 5\\
Netherlands & 5\\
Sweden & 5\\
Australia & 6\\
United Kingdom & 8\\
United States & 9\\
New Zealand & 10\\
\end{tabular}
\end{center}

\subsection{Countries missing}
\label{sec-2-6}

\begin{center}
\begin{tabular}{l}
Country without Public Government\\
\hline
Australia\\
Austria\\
Brazil\\
Chile\\
China (People's Republic of)\\
Colombia\\
Germany\\
Greece\\
Iceland\\
India\\
Indonesia\\
Korea\\
Latvia\\
Mexico\\
New Zealand\\
Russia\\
Slovenia\\
South Africa\\
Switzerland\\
\end{tabular}
\end{center}

\begin{center}
\begin{tabular}{lll}
Country & Annually & Quarterly\\
\hline
Belgium & x & x\\
Canada & x & x\\
Czech Republic & x & x\\
Denmark & x & x\\
Estonia & x & x\\
Finland & x & x\\
France & x & x\\
Hungary & x & x\\
Ireland & x & x\\
Israel & x & \\
Italy & x & \\
Japan & x & x\\
Luxembourg & x & x\\
Netherlands & x & x\\
Norway & x & x\\
Poland & x & x\\
Portugal & x & \\
Slovak Republic & x & \\
Spain & x & \\
Sweden & x & x\\
Turkey & x & \\
United Kingdom & x & x\\
United States & x & x\\
\end{tabular}
\end{center}

\section{{\bfseries\sffamily TODO} }
\label{sec-3}

\begin{itemize}
\item Scale data?
\item Make the approriate transformation
\item Use the data for supplementary variable

\item Compare the models (step, using missing data)
\item Quarterly <-> Check dates
\end{itemize}

\section{Quarterly}
\label{sec-4}

\begin{itemize}
\item No data for population
\item Compare amongst countries with only quaterly data
\end{itemize}

\section{Missing data}
\label{sec-5}

It causes bias.

\section{Robustness}
\label{sec-6}

\begin{itemize}
\item Check table of log population robustness (not automated CountryUSA,
labels are messed up)
\end{itemize}

\section{Government Fiscal Transparency IMF Data}
\label{sec-7}

Constant time

\section{Bibliography}
\label{sec-8}
% Emacs 24.5.1 (Org mode 8.2.10)
\end{document}
