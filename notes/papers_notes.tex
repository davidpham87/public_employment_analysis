\section{Paper notes}\label{paper-notes}

\subsection{It Isn't Just about Greece (James Alt, et
al)}\label{it-isnt-just-about-greece-james-alt-et-al}

Analyzes the political origins of differences in adherence to the fiscal
framework of the EU.

It studies the interaction of opaque budget and fiscal policy for
electoral purposes falsify national statistics
(\texttt{creative accounting}, \texttt{fiscal gimmickry} \footnote{Deviations
  from accepted and expected reporting practices})

Election year obscures statistics. The following explain variations in
outcomes

\begin{itemize}
\itemsep1pt\parskip0pt\parsep0pt
\item
  Domestic institution (budget transparency)
\item
  Politics (elections)
\item
  Economic cycles (recessions)
\end{itemize}

and this enforces the saying

\begin{quote}
Source of fiscal discipline is at the domestic level.
\end{quote}

\emph{Gimmick} is the action of manipulating, embellish facts in order
to alter reality:

\begin{itemize}
\itemsep1pt\parskip0pt\parsep0pt
\item
  Improve appearances of public finance statistics (budget balance,
  debt) without effect of real statistics.
\item
  Asymmetric information in fiscal/economic unions.
\item
  \emph{Misreport} of fiscal data, \emph{hidden actions} (employing
  gimmicks involving creative or unorthodox accounting treatments of
  operations to interpret rules or cheating).
\end{itemize}

Incentives are audiences (voters, bond markets, EU). Three types of
trade-off actions:

\begin{enumerate}
\def\labelenumi{\arabic{enumi}.}
\itemsep1pt\parskip0pt\parsep0pt
\item
  Real adjustement to tax and expenditures, unpopular for voters;
\item
  Do nothing, with penalties from the EU;
\item
  Gimmickries, with intertemporal trade-off, future high bond yields,
  political unrest if discovered.
\end{enumerate}

Fiscal opacity is penalized by bond markets. Strategy: rule violation
and gimmickry with absence of market discipline. The fiscal transparency
rarely evolve through time, and if yes, seldom in the right direction.

\subsubsection{Stock-flow adjustments}\label{stock-flow-adjustments}

\begin{align*}
 SAF=  D_t - D_{t-1} + B_t,
\end{align*}

where \(D_t\) is the debt level at time \(t\) and \(B_t\) is the budget
balance at time \(t\). Critics about this measure: how can we correctly
observe the changes? For example, a decrease of debt of two with a
surplus of the budget balance of 1 has the same statistics as an
increase of debt of 1 and a deficit of the budget balance of 2.

The SFA is decomposed into several components, and two of them are
significant for the study

\begin{itemize}
\item
  \textbf{shares and other equity}: used for translating net cash
  transfer (debt) as share purchasing.
\item
  \textbf{other account payable} (OPA): goods and services that have
  been delivered but not yet paid for. The SFA increases when the OPA
  decreases.
\end{itemize}

\subsubsection{Explaining Gimmicks}\label{explaining-gimmicks}

\begin{itemize}
\itemsep1pt\parskip0pt\parsep0pt
\item
  Transparency diminishes the appeal of gimmicks.
\end{itemize}

Other explanatory variables

\begin{enumerate}
\def\labelenumi{\arabic{enumi}.}
\itemsep1pt\parskip0pt\parsep0pt
\item
  Fiscal rules: if the conditions of the SGP are respected.
\item
  Electoral incentive: year left in the office for the incumbent
  government, the amount of gimmickry should be bigger with fewer years
  left.
\item
  Economics conditions: distinction between fast growth and below-trend
  growth.
\end{enumerate}

\subsection{Redistributive Public
Employment}\label{redistributive-public-employment}

American cities public employment is used for redistributive purposes.
It is a disguises way of channeling resources from middle clas svoter to
disadvantaged citizens when an explicit tax-transfer scheme would not
find political support.

Ethincally fragmented cities tend to have larger public employment.

\begin{quote}
But are the people employed by the government come from the
disadvantaged groups?
\end{quote}

\subsubsection{Theoretical framework}\label{theoretical-framework}

\begin{itemize}
\item
  A two period timeframe, with election after the first period.
\item
  Two classes of voters: \emph{middle} class and the \emph{poor}.
\item
  Two contestants for the government.
\item
  Each contestant need the support of the middle class.
\item
  Define \(B\) as the benefit of a public project (employing the
  \emph{poor} to complete it). Then
  \(B \in \{B_L, B_H\}, 0 < B_L < B_H\) with\\

  \begin{align*}
  B = 
  \left\{
    \begin{array}{ll}
    B_L & \mbox{with probability } 1-\theta \\
    B_H & \mbox{with probability } \theta
    \end{array}
  \right.
  \end{align*}

  with \(\theta\) is random variable taking either \(\theta_L\) or
  \(\theta_H\) with \(0 < \theta_L < \theta_H < 1\). When
  \(\theta=\theta_L\), it is more efficient to make a cash transfer than
  implemeting the project.
\item
  The incumbent government observe the realisation of \(\theta\) before
  deciding to implement a public project or not.
\item
  Two type of contestant: one for the middle class and conduct the
  public project only if \(\theta = \theta_H\), and one supporting the
  poor, implemeting the public project for any value of \(\theta\) if
  the action does not prevent them from winning the next election.
\item
  Voters ignore which type are the politicians, however they have
  perception (priors) about the incumbent and the challenger.
\item
  Depending on the priors of the voters for the incumbent, if he favors
  the poors, he might or not implement the public project even though it
  is not efficient.
\end{itemize}

\subsubsection{Data}\label{data}

All US cities with more than 25 000 inhabitants using official
statistics form the \_City and County Databook, Censur of Governments.

\subsubsection{Regression}\label{regression}

The following formula has been fitted

\begin{align*}
    Y_{i} = \mu + \beta_1 I_i + \beta_3 * X_i + \varepsilon_i         
\end{align*}

where

\begin{itemize}
\itemsep1pt\parskip0pt\parsep0pt
\item
  \(Y_i\) is the government employment per 1000 population, or per 1000
  working age population.
\item
  \(I_i\) is a measure of inequality (gini, mean/median income,
  percentage of person below the poverty level, percentage of families
  below the poverty level).
\item
  \(X_i\) is a data matrix containing the statistics: fraction of 25+
  years old with a university degree, american state, unemployement
  rate, money income per thousand dollars, log of city population, or
  fraction of retired (65+) population and ethincity
  \(1 - \sum_i (ethnic_i)^2\) where \(ethnic_i\) is the share of
  population self-identified with ethnic origins \(i\).
\end{itemize}

\subsubsection{Robustness}\label{robustness}

\begin{itemize}
\itemsep1pt\parskip0pt\parsep0pt
\item
  Taking out the state
\item
  Outliers
\item
  Total public spending per capita
\item
  Checking the coefficient with and without inequality
\item
  Fragmentation of type of public employment (central administration,
  streets and highway, housing and community developement, libraries,
  natural sciences, parks and recreation, sewerage, and solide waste
  management.
\end{itemize}

\subsection{Fiscal Transparency and Public
Employment}\label{fiscal-transparency-and-public-employment}

\subsubsection{Framework}\label{framework}

Incumbent government prefer to stay in the government even though they
don't take the right decision. Public employment is more popular than
wealth transfer and tax cut.

With windfall from gdp growth, country with low fiscal transparency will
increase their public employment.

\subsubsection{Control variable
economic}\label{control-variable-economic}

\begin{itemize}
\itemsep1pt\parskip0pt\parsep0pt
\item
  OECD countries between 1996 to 2010
\item
  GDP per capita in constant price in order to control for the Wagner's
  Law
\item
  Unemployment rate
\item
  Government spending as percent of GDP
\item
  Left or right party
\item
  Election year
\item
  Country fixed effect
\end{itemize}

The formula is given by, for country \(i\) and time \(t\):

\begin{align*}
Y_{it} = \alpha + \beta_1 G_{it} + \beta_2 G_{it} T_i + \beta_3 X_{it} + \eta_i + \tau_t + \varepsilon_{it},
\end{align*}

where the variables are

\begin{itemize}
\itemsep1pt\parskip0pt\parsep0pt
\item
  \(Y_{it}\), the public employment;
\item
  \(G_{it}\), GDP growth;
\item
  \(T_{i}\), fiscal transparency;
\item
  \(X\) is the vector of control variables;
\item
  \(\eta_i\), country fixed effect;
\item
  \(\tau_t\), year fixed effect;
\end{itemize}

Public employment also includes employee from government owned companies
and is defined as the ratio of people employed in the government and
these companies over the total work force.

\subsubsection{Robustness analysis}\label{robustness-analysis}

\begin{itemize}
\itemsep1pt\parskip0pt\parsep0pt
\item
  The demographic is used as a control variable.
\item
  Different index of sical transparency IMF's Reports on the Observance
  of Standards and Codes. Average of public information,
  budgetaryprocess, assurance of integrity. With this robustness
  methods, coefficient of \(T_{it}\) and \(T_{it}G_{it}\) are only
  significant at the \(0.1\) level.
\item
  Exclusion of the beginning and ending of the period. Excluding Greece
  and New Zealand.
\end{itemize}

\subsection{Retreat of the state from entrepreneurial
activities}\label{retreat-of-the-state-from-entrepreneurial-activities}

This paper describes the evolution of privatization, deregulation in
network-based service, and the cutback of subsidies in 20 OECD
countries. Our interest lies in the evolution of the employment index
used in the privatization section.

\subsubsection{Relevant statistics}\label{relevant-statistics}

The paper use the ratio between the the number of employed persons by
the public government over total employment. The former is computed as a
weighted sum of employee between the following bodies:

\begin{itemize}
\itemsep1pt\parskip0pt\parsep0pt
\item
  \textbf{Departmental Agencies (DA)}: public administrative bodies
  without their legal identity;
\item
  \textbf{Public Corporations (PC)}: firms that are totally owned by the
  state but have a public legal body. These have a weight
  \(\alpha = 0.75\) in the paper;
\item
  \textbf{State Companies (SC) and Private Firms (PF)}: \emph{SC} are
  \emph{PC} which the states do not hold 100 percent of the shares. For
  \emph{SC} and \emph{PF}, the weight of their number of employee is
  provided by \(\beta \gamma\) where \(\gamma\) is the percentage of
  public owned shares of the entreprise, and \(\beta\) is set at
  \(0.5\).
\end{itemize}

Missing value are interpolated when necessary and for small firms only
there is a cutoff of the 60\% smallest firms.
