\documentclass[]{article}
\usepackage{lmodern}
\usepackage{amssymb,amsmath}
\usepackage{ifxetex,ifluatex}
\usepackage{fixltx2e} % provides \textsubscript
\ifnum 0\ifxetex 1\fi\ifluatex 1\fi=0 % if pdftex
  \usepackage[T1]{fontenc}
  \usepackage[utf8]{inputenc}
\else % if luatex or xelatex
  \ifxetex
    \usepackage{mathspec}
  \else
    \usepackage{fontspec}
  \fi
  \defaultfontfeatures{Ligatures=TeX,Scale=MatchLowercase}
  \newcommand{\euro}{€}
\fi
% use upquote if available, for straight quotes in verbatim environments
\IfFileExists{upquote.sty}{\usepackage{upquote}}{}
% use microtype if available
\IfFileExists{microtype.sty}{%
\usepackage{microtype}
\UseMicrotypeSet[protrusion]{basicmath} % disable protrusion for tt fonts
}{}
\usepackage[margin=3cm]{geometry}
\usepackage[colorlinks=true,urlcolor=blue]{hyperref}
\PassOptionsToPackage{usenames,dvipsnames}{color} % color is loaded by hyperref
\hypersetup{unicode=true,
            pdftitle={Analysis of Public Employment Rate},
            pdfauthor={David Pham},
            pdfborder={0 0 0},
            breaklinks=true}
\urlstyle{same}  % don't use monospace font for urls
\usepackage{longtable,booktabs}
\setlength{\parindent}{0pt}
\setlength{\parskip}{6pt plus 2pt minus 1pt}
\setlength{\emergencystretch}{3em}  % prevent overfull lines
\providecommand{\tightlist}{%
  \setlength{\itemsep}{0pt}\setlength{\parskip}{0pt}}
\setcounter{secnumdepth}{5}

\title{Analysis of Public Employment Rate ($2^{nd}$)}
\author{David Pham}
\date{2016-01-13}

\usepackage[style=alphabetic,backend=biber]{biblatex}
\bibliography{biblio.bib}

% Redefines (sub)paragraphs to behave more like sections
\ifx\paragraph\undefined\else
\let\oldparagraph\paragraph
\renewcommand{\paragraph}[1]{\oldparagraph{#1}\mbox{}}
\fi
\ifx\subparagraph\undefined\else
\let\oldsubparagraph\subparagraph
\renewcommand{\subparagraph}[1]{\oldsubparagraph{#1}\mbox{}}
\fi

\begin{document}
\maketitle

\section{Summary}\label{summary}

In order to study quarterly data, precise election data had to be
retrieved for several countries and some time series had to be updated.
Two methodologies have been applied to find structural breaks, but their
results do not provide clear signal for the analysis, due to the lack of
variability of the public employment rate. Finally, a baseline model is
proposed for continuing the analysis.

\section{Data}\label{data}

Some amount of work has been invested to get precise quarterly data. The
exact election date were retrieved from wikipedia, and the time until
the next election has a quarterly frequency. The time series from
government political orientation and government fractionalization are
updated as well to match the election dates.

The working document can be found
\href{https://github.com/davidpham87/public_employment_analysis/blob/master/PublicEmploymentAnalysis/data/manual_data_entry/transform_yearly_to_quarterly_elections.org}{here}
and the resulting csv on
\href{https://github.com/davidpham87/public_employment_analysis/blob/master/PublicEmploymentAnalysis/data/execrlc_govfrac_yrcurnt_quartery_cleaned.csv}{github.com}
as well.

\section{Bai and Perron structural
breaks}\label{bai-and-perron-structural-breaks}

Two methodology has been applied to detect structural break changes.
However, they do not always coincide and the break are not common to all
countries. There are some liberty and variability of the location of the
changes depending on the tuning parameters (size of clusters).

\section{Impact of quarterly data for the data
analysis}\label{impact-of-quarterly-data-for-the-data-analysis}

The variability of \texttt{egrlf} is so small that linear regression has
an almost perfect fit by only using country dummy variables. Hence, it
is compulsory to analyze the difference of \texttt{egrlf} and to
understand which variables can explain the variability. However, this
methods and the frequency of the data greatly diminish the signal to
noise ratio (the adjusted r squared is at ten percent at best).

\section{Union of explanatory and control
variables}\label{union-of-explanatory-and-control-variables}

\cite{aaskoven2015fiscal} and \cite{alt2014isn} use the following
variables as control variables:

\begin{longtable}[c]{@{}lll@{}}
\toprule
Variable & \cite{aaskoven2015fiscal} &
\cite{alesina2000redistributive}\tabularnewline
\midrule
\endhead
GDP Growth & x &\tabularnewline
GDP per capita & x &\tabularnewline
Money income & & x\tabularnewline
Unemployment rate & x & x\tabularnewline
Government spending & x & x\tabularnewline
Log of population & & x\tabularnewline
\bottomrule
\end{longtable}

The next variable are used to explain the variation of \texttt{egrlf}:

\begin{longtable}[c]{@{}llll@{}}
\toprule
Variable & \cite{aaskoven2015fiscal} & \cite{alesina2000redistributive}
& \cite{alt2014isn}\tabularnewline
\midrule
\endhead
Fiscal transparency & x & &\tabularnewline
Government political orientation & x & &\tabularnewline
Years until next election & x & & x\tabularnewline
Measures of inequality & & x &\tabularnewline
Fiscal rules & & x & x\tabularnewline
Potential GDP gap (Boom and Slump) & & & x\tabularnewline
\bottomrule
\end{longtable}

From the above tables and some regression, for quarterly data the
following baseline model is proposed. The dependent variable is the
change in \texttt{egrlf} and the control variables are

\begin{itemize}
\tightlist
\item
  Autocorrelation with lag one.
\item
  GDP growth, QoQ, lagged one quarter;
\item
  Change in unemployment rate (absolute difference), QoQ;
\item
  Government expenditure (GEXP) in percent of GDP (interpolated from
  annual data);
\item
  Changes in GEXP;
\item
  Log of adult population;
\item
  Change in GDP per capita in USD, QoQ.
\end{itemize}

\printbibliography

\end{document}

% Local Variables:
% TeX-engine: xetex
% End:
