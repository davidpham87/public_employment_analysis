\chapter{Conclusion}

This semester paper quickly summarizes the current literature on the theory of
civil servants and tries to find a reasonable model which could explain
quarterly variation of the share of public employees in the labor force. With
analysis on annual data, it is believed that the level of wealth, the fiscal
transparency, the government political partisanship, the proximity of election
terms or the degree of inequality would correlate with the public employment
rate. With quarterly frequency, data suggest that the influence of these
variables on the public employment rate might be weak. Even though our analysis
showed some significant unadjusted $p$-values with these variables, these
become statistically insignificant after correction. These results do not
support the conclusions of the existing literature. Nevertheless, one can not
exclude that the adjustment might have been too sharp or that some observations
of the data were imprecise.

In order to test these ideas with quarterly frequency, the OECD Economic
Outlook quarterly data set has been retrieved and many missing variables were
interpolated from annual data from the same data set or alternative sources,
leading to potential mistakes.

In order to confirm or invalidate this conclusion, one could replicate the
study in order to check the consistency of the data, restrict the number
of variables at the beginning of the study in order to increase the power of the
statistical test, and maybe use more precise methods for the interpolation in
order to capture undetected signals. The challenge of over-fitting and
overstating statistical significance remains and one should probably stick to
simple statistical models.

In order to ease the results, the data and programming scripts are shared on
Github\footnote{\url{https://github.com/davidpham87/public_employment_analysis}}.
References should provide enough indication concerning the technical and the
economical background.

Finally, the base model with unemployment rate, government revenues and net
lending seems to be surprisingly consistent to explain public employment share
with quarterly data from the OECD. This is a deceptive result as it is
difficult to see why these variables would be correlated with the share of
civil servants according to the literature.


%%% Local Variables: ***
%%% mode:latex ***
%%% TeX-master: "semester_paper_sfs.tex"  ***
%%% End: ***
%%% reftex-default-bibliography: ("biblio.bib")
