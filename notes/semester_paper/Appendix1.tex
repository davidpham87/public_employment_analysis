\chapter{R Implementation Details}
\label{app:complement}

The complete \textsf{R} code to generate the result is stored on
\url{https://github.com/davidpham87/ethz\_missing\_data} where the procedure
are well documented.

\section{Completion of the original FLAS data set}

One of the critical element in the comparison of completion methods is the
complete data set from which the artificial incomplete data set is derived. As
explained, the original FLAS data set contains missing value and the following
steps to complete it are described below.

For each of the \texttt{mice} and \texttt{mi} packages, 20 imputed data set are
created. The selected final observation is then either the arithmetic average
for numerical variables or the mode for factor variables among the 40 imputed
data set and it becomes the baseline for all the comparison.

\lstinputlisting[language=R, style=Rstyle]{../../R/flas_completion.R}

\section{Generation of low-level errors}

If the reader is interested in looking at the output of these script, they are
available on
\url{github.com/davidpham87/ethz_missing_data/tree/master/R/bug_replication}.

\paragraph{Amelia}

\texttt{Amelia}'s implementation seems to have difficulties to handle some
structures of data matrix with missing values. The following code snippet shows
how to reproduce this errors.

\lstinputlisting[language=R, style=Rstyle]{../../R/bug_replication/bug_amelia.R}

\paragraph{Impute}

The \texttt{impute} package seems to call an underlying \texttt{fortran}
procedure. The call can throw some segmentation fault errors depending on
the nature of some arguments.

\lstinputlisting[language=R, style=Rstyle]{../../R/bug_replication/bug_impute_knn.R}

%%% Local Variables:
%%% mode: latex
%%% TeX-master: "semester_paper_sfs"
%%% End:
