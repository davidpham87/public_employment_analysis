\chapter{Introduction}

Since the start of the financial crisis of the 2008, countries and governments
heavily relied on their respective central banks to boost their economies and
support growth. Nevertheless, governments and their leaders also have the
responsibility to provide employment to the population. They can use several
channels to reach their end: low corruption, employment friendly laws and
naturally employ professional directly, the so-called public servants. Public
employment is main contributor to GDP through government expenditures

The share of public servants compared to total labor force varies over time,
and several hypothesis exist to explain why. One of the recurring idea is that
incumbent governments have incentives to increase the number of civil servant
before elections to raise their chances of reelection.

The purpose of this semester paper is to gather synthesize the current opinion
of the subject, and most importantly tried to replicate the results by using
quarterly data from the OECD Economic Outlook.

The added value of this work is that it is probably the first time that the
quarterly data set is used to study the problem. The main challenge to complete
the work has been to gather the data, tidy them, merge them. Moreover,
as the data set contains hundreds of variables and is observational,
statistical analysis should be executed cautiously.

The work is hopefully as reproducible as possible. The github
repository\footnote{\url{https://github.com/davidpham87/public_employment_analysis}}
of the project offers full transparency over the data and the code used to
handle them.

The structure of the semester paper reflects the above commentary: a first part
is devoted to summarize the research of recent papers, then the second part
focuses on the statistical analysis of the data.

%%% Local Variables:
%%% mode: latex
%%% TeX-master: "MasterThesisSfS"
%%% End:
