\chapter{Introduction}

Since the start of the 2008 financial crisis, countries and governments heavily
have relied on their respective central banks to boost their economies and
support growth. Thanks to low interest rates, governments and their leaders
could probably invest in the economy and provide employment to the
population. They can provide it by several means: keeping a low corruption,
creating an employment friendly environment and naturally employ professionals
directly, the so-called public servants.

The share of civil servants with respect to total labor force varies over time,
and several hypothesis exist to explain why. One of the recurring idea is that
incumbent governments have incentives to increase the number of civil servants
before elections to raise their chances of reelection and this would be easier
in environment where fiscal transparency is low.

The purpose of this semester paper is to synthesize the current literature of
the subject, and most importantly to gather and to prepare data for a
replication of the results of the literature by using quarterly data
from the OECD Economic Outlook.

The added value of this work is that a quarterly data set is used as input to
study the problem. This would be for the first time, to the best of our
knowledge. The main challenge to complete the work has been to gather the data,
to tidy them and to merge them. Moreover, as the data set contains hundreds of
observational variables, statistical analysis should be executed cautiously.

With modern technology, studies should be as transparent and reproducible as
possible. Hopefully, the Github
repository\footnote{\url{https://github.com/davidpham87/public_employment_analysis}}
of the project offers full transparency over the data and the code used to
handle them. The analysis is performed with the statistical environment
\textsf{R} (\cite{R2015}) and the main scripts to perform the analysis is supplemented in the
appendix.

The structure of the semester paper is following: a first part is devoted to
summarize the research of recent papers, then it focuses on the statistical
analysis of the data, before concluding.

%%% Local Variables:
%%% mode: latex
%%% TeX-master: "semester_paper_sfs"
%%% End:
